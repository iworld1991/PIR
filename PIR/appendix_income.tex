\subsubsection{PR by realized earnings}

Standard models with idiosyncratic income risks do not assume heterogeneity by permanent income in addition to the observed group factors that may affect permanent income, such as education. Is it so in risk perceptions? It turns out that PR does correlate with the realized outcomes of the individuals. For a subsample of around 4000 observations, SCE surveys the annual earning of the respondent along with their risk perceptions. I group individuals into 10 groups based on their reported earning (within the same time) and plot the average risk perceptions against the decile rank in Figure \ref{fig:barplot_byinc}. Perceived risks decline as one's earnings increase.  This is not exactly consistent with the uptick in income risks for the highest income group, as documented by \cite{bloom2018great} using tax records of income. The most likely explanation is that the small sample I used from SCE does not cover actual top earners. The average annual earning of the top income group is between \$45,000 and \$120,000 in our sample.  

\begin{figure}[!ht]
    	\caption{Perceived Wage Risks by Earning Decile}
    	\label{fig:barplot_byinc}
    	\begin{center}\adjustimage{max size={0.7\linewidth}}{figures/boxplot_rvar_earning.png}\end{center}
    \begin{flushleft}Note: this figure plots average perceived income risks by the decile of annual earning of the same individual.\end{flushleft}
    \end{figure}