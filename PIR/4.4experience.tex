\hypertarget{experiences-and-perceived-risk}{%
\subsection{Experiences and perceived
risk}\label{experiences-and-perceived-risk}}
%%%%%%%%%%%%%%%%%%%%
%% need to add extrapolation 
%%%%%%%%%%%%%%%%%%%%%%%%%%%%%%




\begin{center}
[TABLE \ref{extrapolation} HERE]
\end{center}

%%%%%%%%%%%%%%%%%%%%%%%%%%%%%%%%%%%%%%
%%%. ADD MORE HERE 
%%%%%%%%%%%%%%%%%%%%%%%%%%%%%%%%%%%%


Different generations also have different perceived income risks. Let us
explore to what extent the cohort-specific risk perceptions are
influenced by the income volatility experienced by that particular
cohort. Different cohorts usually have experienced distinct
macroeconomic and individual histories. On one hand, these non-identical
experiences could lead to long-lasting differences in realized life-long
outcomes. An example is that college graduates graduating during
recessions have lower life-long income than others.
(\cite{kahn2010long}, \cite{oreopoulos2012short},
\cite{schwandt2019unlucky}). On the other hand, experiences may have
also shaped people's expectations directly, leading to behavioral
heterogeneity across cohorts (\cite{malmendier2015learning}). Benefiting
from having direct access to the subjective income risk perceptions, I
could directly examine the relationship between experiences and
perceptions.

Individuals from each cohort are borned in the same year and obtained
the same level of their respective highest education. The experienced
volatility specific to a certain cohort \(c\) at a given time \(t\) can
be approximated as the average squared residuals from an income
regression based on the historical sample only available to the cohort's
life time. This is approximately the unexpected income changes of each
person in the sample. I use the labor income panel data from PSID to
estimate the income shocks.
\footnote{I obtain the labor income records of all household heads between 1970-2017. Farm workers, youth and olds and observations with empty entries of major demographic variables are dropped. }
In particular, I first undertake a Mincer-style regression using major
demographic variables as regressors, including age, age polynomials,
education, gender and time-fixed effect. Then, for each cohort-time
sample, the regression mean-squared error (RMSE) is used as the
approximate to the cohort/time-specific income volatility.

There are two issues associated with such an approximation of
experienced volatility. First, I, as an economist with $PSID$ data in my
hand, am obviously equipped with a much larger sample than the sample
size facing an individual that may have entered her experience. Since
larger sample also results in a smaller RMSE, my approximation might be
smaller than the real experienced volatility. Second, however, the
counteracting effect comes from the superior information problem,
i.e.~the information set held by earners in the sample contains what is
not available to econometricians. Therefore, not all known factors
predictable by the individual are used as a regressor. This will bias
upward the estimated experienced volatility. Despite these concerns, my
method serves as a feasible approximation sufficient for my purpose
here.

The right figure in Figure \ref{fig:var_experience_data} plots the
(logged) average perceived risk from each cohort \(c\) at year \(t\)
against the (logged) experienced volatility estimated from above. It
shows a clear positive correlation between the two, which suggests that
cohorts who have experienced higher income volatility also perceived
future income to be riskier. The results are reconfirmed in Table
\ref{micro_reg}, for which I run a regression of logged perceived risks
of each individual in SCE on the logged experienced volatility specific
to her cohort while controlling individuals age, income, educations,
etc. What is interesting is that the coefficient of \(expvol\) declines
from 0.73 to 0.41 when controlling the age effect because that
variations in experienced volatility are indeed partly from age
differences. While controlling more individual factors, the effect of
the experienced volatility becomes even stronger. This implies potential
heterogeneity as to how experience was translated into perceived risks.

How does experienced income shock per se affect risk perceptions? We can
also explore the question by approximating experienced income growth as
the growth in unexplained residuals. As shown in the left figure of
Figure \ref{fig:var_experience_data}, it turns out that a better
past labor market outcome experienced by the cohort is associated with
lower risk perceptions. This indicates that it is not just the
volatility, but also the change in level of the income, that is
asymmetrically extrapolated into their perceptions of risks.

\begin{center}
[FIGURE \ref{fig:var_experience_data} HERE]
\end{center}


\begin{comment}

%%%%%%%%%%%%%%%%%%%%%%%%%%%%
%%% Need to rewrite 
%%%%%%%%%%%%%%%%%%%%%%%%%%%%

In theory, individual income change is driven by both aggregate and
idiosyncratic risks. It is thus worth examining how experienced outcome
from the two respective source translate into risk perceptions
differently. In order to do so, we need to approximate idiosyncratic and
aggregate experiences, separately. The former is basically the
unexplained income residual from a regression controlling time fixed
effect and also time-education effect. Since the two effects pick up the
sample-wide or group-wide common factors of each calendar year, it
excludes aggregate income shocks. The difference between such a residual
and one from a regression dropping the two effects can be used to
approximate aggregate shocks. As an alternative measure of aggregate
economy, I use the official unemployment rate. For all aggregate
measures, the volatility is correspondingly computed as the variance
across time periods specific to each cohort.





Figure \ref{fig:experience_id_ag_data} plot income risk perceptions against both aggregate and idiosyncratic experiences measured by the level and the volatility of shocks. It suggests different patterns
between the aggregate and idiosyncratic experiences. In particular, a
positive aggregate shock (both indicated by a higher aggregate income
growth, or a lower unemployment rate) is associated with lower risk
perceptions. Such a negative relationship seems to be non-existent at
the individual level. What's common between aggregate and idiosyncratic
risks is that the volatility of both kinds of experiences are positively
correlated with risk perceptions. Such correlations are confirmed in a
regression of controlling other individual characteristics, as shown in
Table \ref{micro_reg}. Individual volatility, aggregate volatility and
experience in unemployment rates are all significantly positively
correlated with income risk perceptions.


\end{comment}

%%%%%%%%%%%%%%%%%%%%%%%%%%%%%%%%%%%%%%%%%%%%%%%%%
%%% table that reports the recent experience and perceived risks 
%%%%%%%%%%%%%%%%%%%%%%%%%%%%%%%%%%%%%%%%%%%%%%%

%% earning volatility to perceived risks 
%% unemployment experience to perceived risks 