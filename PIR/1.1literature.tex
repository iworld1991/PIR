
\hypertarget{related-literature}{%
\subsection{Related literature}\label{related-literature}}

First, this paper closely builds on the literature estimating both cross-sectional and time trends of labor income risks and partial insurance. Early work such as \cite{abowd1989covariance,gottschalk1994growth, carroll1997nature} started the standard practice in the literature of estimating income risks by decomposing it into components of varying persistence based on panel data. Subsequent work explores time-varying and macro trend of idiosyncratic income risks. For instance, \cite{meghir2004income} allows for time-varying risks or conditional heteroscedasticity in the traditional permanent-transitory model. \cite{blundell_consumption_2008} uses the same specification of income process to estimate partial insurance in conjunction with consumption data. More recently,  \cite{bloom2018great} found the idiosyncratic income risks have declined in recent decades.\footnote{Although  \cite{moffitt2020reconciling} found no such obvious trend for the same period by synthesizing various data sources.} Moreover, recent evidence relied upon detailed administrative records and larger data samples highlight the asymmetry and cyclical behaviors of idiosyncratic earning/income risks \citep{storesletten2004cyclical, guvenen2014nature,arellano2017earnings,guvenen2019data,bayer2019precautionary,guvenen2021data}. Besides, a separate literature focus on job-separation and unemployment risks \citep{low2010wage,davis2011recessions}. In the Appendix, Table \ref{tab:risks_literature} summarize the income process and estimated risks in a number of selected papers in this literature. Compared to this work, the novelty of this paper lies in the focus on the subjective perceptions of labor risks and how it is correlated with the realized income risks estimated from the income panel. 

Closely related is the well-documented issue in the partial insurance literature: ``insurance or information''
(\cite{pistaferri_superior_2001},
\cite{kaufmann_disentangling_2009}, \cite{meghir2011earnings}, \cite{kaplan2010much}). In any
empirical tests of consumption insurance or consumption response to
income, there is always a concern that what is interpreted as the shock
has actually already entered the agents' information set. On the flip side, agents may not instantaneously incorporate the innovations to income and the macroeconomy, although economists assume so, leading to the excessive smoothness of supposedly anticipated shocks (\cite{flavin_excess_1988}). My paper shares a similar spirit with these studies in the sense that I try to tackle the identification problem in the same approach:\footnote{Recently, in the New York Fed \href{https://libertystreeteconomics.newyorkfed.org/2017/11/understanding-permanent-and-temporary-income-shocks.html}{blog}, the authors followed a similar approach to decompose the permanent and transitory shocks.} directly using the expectation data and explicitly controlling for what are truly conditional expectations of the agents. This helps economists avoid making assumptions about what is exactly in the agents' information set. What differentiates my work from other authors is that I directly use survey-reported income risks, which are available from density forecasts, instead of estimated risks using the difference between expectation and realization. One advantage of the approach in this paper is that I can directly study individual-specific risks instead of groups.

Third, the paper
speaks to an old but recently reviving literature of studying
consumption/saving behaviors in models incorporating imperfect
expectations and perceptions. For instance, \cite{pischke1995individual} explores the implications of the
incomplete information about aggregate/individual income innovations by modeling agent's learning about income component as a signal extraction problem. \cite{wang2004precautionary} extends the framework to incorporate precautionary saving motives. In a similar spirit, \cite{carroll_sticky_2018} reconciles the low micro-MPC and high macro-MPCs by introducing to the model an information rigidity of households in learning about macro news while being updated about micro news. \cite{rozsypal_overpersistence_2017} found that households' expectation of income exhibits an over-persistent bias, which may explain high MPCs out of transitory income shocks. More recently, \cite{broer2021information} incorporates information choice in a standard consumption/saving model to explore its implication for wealthy inequality. My paper has a similar flavor to all of these studies in that it also emphasizes the role of perceptions. But it has two major distinctions. First, this paper focuses on the second moment, namely income risks. Second, although most of this existing work explicitly specifies a mechanism of expectation formation deviating from the full-information-rational-expectation benchmark, this paper advocates for disciplining the model assumptions regarding belief heterogeneity by directly using survey data. \footnote{See \cite{bhandari2019survey} for another example of directly using survey data to discipline subjective beliefs in standard macro models.}

Besides, the paper is indirectly related to the research that advocated
for eliciting probabilistic questions measuring subjective uncertainty
in economic surveys (\cite{manski_measuring_2004},
\cite{delavande2011measuring}, \cite{manski_survey_2018}). Although the
initial suspicion concerning to people's ability in understanding, using
and answering probabilistic questions is understandable,
\cite{bertrand_people_2001} and other works have shown respondents have
the consistent ability and willingness to assign a probability (or
``percent chance'') to future events. \cite{armantier_overview_2017}
have a thorough discussion on designing, experimenting and implementing
the consumer expectation surveys to ensure the quality of the responses.
Broadly speaking, the advocates have argued that going beyond the
revealed preference approach, availability to survey data provides
economists with direct information on agents' expectations and helps
avoids imposing arbitrary assumptions. This insight holds for not only
point forecast but also and even more importantly, for uncertainty,
because for any economic decision made by a risk-averse agent, not only
the expectation but also the perceived risks matter a great deal.

Finally, empirically, this paper is related to the literature studying expectation formation using subjective surveys. There has been a long list of ``irrational expectation'' theories developed in recent decades on how agents deviate from full-information rationality benchmark, such as sticky expectation, noisy signal extraction, least-square learning, etc. Also, empirical work has been devoted to testing these theories in a comparable manner (\cite{coibion2012can},
\cite{fuhrer2018intrinsic}). But it is fair to say that thus far,
relatively little work has been done on individual variables such as labor income, which may well be more relevant to individual economic decisions. Therefore, understanding expectation formation of the individual variables, in particular, concerning both mean and higher moments, will prove fruitful for macroeconomic modelings.