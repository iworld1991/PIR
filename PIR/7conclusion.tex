\hypertarget{conclusion}{%
\section{Conclusion}\label{conclusion}}

Incomplete-market macroeconomic models that admit uninsured idiosyncratic income risks have become the new paradigm of macroeconomic analysis in the past decade. This paper builds on this vibrant literature by relaxing the long maintained assumption in these models that agents perfectly understand the underlying size of income risks and behave optimally accordingly. 

Utilizing the recently available large-scale survey data that elicits density forecast of earnings and labor outcome, I incorporate salient empirical patterns of income risk perceptions such as heterogeneity, extrapolation and state-dependence in these models. The survey evidence also indicates the possible ``superior information'' problem documented in the literature, confirming an upward bias in the assumed size of income risks in these models compared to what people report in surveys. This  helps explain the low liquid asset holdings of a large fraction of consumers, or the presence of many hands-to-mouth consumers. 

This paper also presents a demonstration of the rich possibility of incorporating survey data reflecting real-time heterogeneity in expectations/perceptions in heterogeneous-agent models. In a world with increasingly rich survey data that directly measures expectations, economists are no longer forced to make stringent  assumption of rational expectations. Directly using survey-implied heterogeneity helps match empirical patterns of the macroeconomy. 