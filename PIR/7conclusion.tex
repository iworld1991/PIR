\hypertarget{conclusion}{%
\section{Conclusion}\label{conclusion}}

Incomplete-market macroeconomic models that admit uninsured idiosyncratic income risks have become the new paradigm of macroeconomic analysis in the past decade. 


Utilizing the recently available large-scale survey data that elicits density forecasts of wage growth, I incorporate salient empirical patterns of income risk perceptions such as heterogeneity, extrapolation and state-dependence in these models. The survey evidence indicates the possible ``unobserved heterogeneity'' or the ``superior information'' problem documented in the literature, confirming an upward bias in the assumed size of income risks in these models compared to what people report in surveys. Incorporating the survey-implied heterogeneity and lower perceived risks helps partly explain the low liquid asset holdings of a large fraction of households, the presence of many hands-to-mouth agents and an additional fraction of wealthy inequality. 

As an additional exploration, I also extend the benchmark model to allow for subjective perceived risks to be different from the objective ones that drive the realized dispersion of the shocks, possibly due to behavioral bias. I explore the effects of such a wedge on ex-ante self-insurance motives and ex post realized income/wealth inequality.  

This paper also presents a demonstration of the rich possibility of incorporating survey data reflecting real-time heterogeneity in expectations/perceptions in heterogeneous-agent models. In a world with increasingly rich survey data that directly measures expectations, economists are no longer forced to make stringent  assumption of rational expectations. Directly using survey-implied heterogeneity helps match empirical patterns of the macroeconomy. 
