
\hypertarget{cross-sectional-heterogeneity}{%
\subsection{Observable and unobservable
heterogeneity}\label{cross-sectional-heterogeneity}}

%This section inspects some basic cross-sectional patterns of the perceived income risks. 
In both income risk estimation and parameterization of the standard incomplete market macro models, it is common practice to assume idiosyncratic risks differ by certain observable factors such as education, gender, and age, and furthermore, there is no additional within-group heterogeneity in the degree of the risks.\footnote{For instance, \cite{meghir2004income} found that more educated workers are faced with higher income risks than the less-education ones. In addition, \cite{sabelhaus2010great, bloom2018great} documented that income risks decrease with age, and vary with the current income level in a non-monotonic U-shape. \cite{cagetti2003wealth},\cite{blundell_consumption_2008}, \cite{carroll2017distribution} allows for heterogeneous risks across different demographic variables in their models.} This section reports the finding that although the observed heterogeneity in PR across individuals does reflect between-group differences along dimensions economists have commonly assumed, a dominant fraction of the differences in PR is attributed to other unobservable heterogeneity. Furthermore, even in those observable dimensions, the group heterogeneity seen in PR does not coincide with that seen in estimated risks.


Figure
\ref{fig:group_compare} plots the group average of perceived risk, approximated wage volatility $Var(\Delta \hat e)$ as defined in Equation \ref{Eq:definition_volatility}, and the estimated risk $Var_t(\Delta \hat e)$ by age, gender and education. As to the education-risk profile, both wage volatility and approximated risks are higher for more educated workers. This is consistent with the finding of \cite{meghir2004income} using labor income instead of wage. In contrast, risk perceptions exhibits an opposite pattern with respect to education level, in that less-educated workers perceive wage risks to be higher than more-educated workers. As to the life-cycle pattern of risks, neither wage volatility nor estimated risks shows very clear heterogeneity across different ages.\footnote{The homogenous age pattern of wage risks is not necessarily contradictory with the well documented declining pattern estimated using data on household income or total earning\footnote{For instance, \cite{cagetti2003wealth}, \cite{sabelhaus2010great}, etc.}. It is likely that the decline of income risks over the life cycle has to do with non-wage risks or better insurance via work arrangements over the life cycle. } In contrast, perceived risks nearly monotonically declines over the life cycle for both males and females. These findings are also confirmed in Table \ref{tab:sipp_sum_stat}, which reports the group average of PR, wage volatility and estimated risks. 

Another salient fact is that PR is always smaller than estimated risks. In particular, both wage volatility and estimated risk of different groups fall in the range of 5-15\% per year (in standard deviation), which broadly aligns with the estimates in a large literature and that used in models, as summarized in Table \ref{tab:risks_literature}. But the average perceived risks reported in the survey is only around 3-4\%, at least half smaller. For instance, a male high school graduate on average perceives annual wage risk to be 4 percentage points in terms of standard deviation, while the income risk implied by wage panel data for the same group is above 9-10 percentage points. 

%It is clear that the subjective risk perceptions decline over the life cycle, consistent with the estimated risk from realizations of income. It is important to notice, however, in principle, the reasons for which subjective risk perceptions decline as one age may not be exactly the same as the one for the same pattern of the actual profile. For instance,  as one accumulates experience over time, it may also reduce the subjective uncertainty about the income dynamics of themselves. 

\begin{center}
 [FIGURE \ref{fig:group_compare} HERE]
 \end{center}
 
 
Standard models with idiosyncratic income risks do not assume heterogeneity by permanent income in addition to the observed group factors that may affect permanent income, such as education. Is it so in risk perceptions? It turns out that PR does correlate with the realized outcomes of the individuals. For a subsample of around 4000 observations, SCE surveys the annual earning of the respondent along with their risk perceptions. I group individuals into 10 groups based on their reported earning (within the same time) and plot the average risk perceptions against the decile rank in Figure \ref{fig:barplot_byinc}. Perceived risks decline as one's earnings increase.  This is not exactly consistent with the uptick in income risks for the highest income group, as documented by \cite{bloom2018great} using tax records of income. The most likely explanation is that the small sample I used from SCE does not cover actual top earners. The average annual earning of the top income group is between \$45,000 and \$120,000 in our sample.  

\begin{center}
 [FIGURE \ref{fig:barplot_byinc} HERE]
\end{center}

But despite controlling for all the observable factors of the individuals, there remains a large degree of heterogeneity that seems to be most likely attributable to other unobserved factors. Figure \ref{fig:histmoms} plots the
cross-sectional distribution of unexplained residuals of perceived income risks both in nominal and real terms after controlling for observable individual characteristics including age, age polynomial, gender, education, type of work, and time fixed effects, respectively. Controlling for the time fixed effects is important because the focus here is on the idiosyncratic risks perceived by agents. 

In both nominal
and real terms, the distribution is right-skewed with a long tail.
Specifically, most of the workers have perceived a standard deviation of nominal
earning growth ranging from zero to \(4\%\) wage growth a year). But in the tail, some
of the workers perceive risks to be as high as \(7-8\%\) standard
deviation a year. To have a better sense of how large the risk is,
consider a median individual in our sample, who has an expected wage 
growth of \(2.4\%\), and a perceived risk of \(1\%\) standard deviation.
This implies by no means negligible earning risk.
\footnote{In the appendix, I also include histograms of expected income growth and subjective skewness, which show intuitive patterns such as nominal rigidity. Besides, about half of the sample exhibits non-zero skewness in their subjective distribution, indicating asymmetric upper/lower tail risks.}

Besides observable factors controlled in the first-step regression, individual fixed effects are important in explaining the heterogeneity in PRs. In particular, the $R^2$ of the regression without individual fixed effects is at most 10\%, while including fixed effects increases $R^2$ to $70\%$. This finding has two implications. First, the role of unobservable heterogeneity seems to suggest that the conventional practice of estimating and modeling income risks differently by demographic groups has limitations. Second, the survey-implied heterogeneity in PR can be directly put into use to model heterogeneous income risks without requiring a strong stance on the explanations of the source of heterogeneity.  

\begin{center}
[FIGURE \ref{fig:histmoms} HERE]
\end{center}

