
    \hypertarget{data-variables-and-density-estimation}{%
\section{Data, variables and density
estimation}\label{data-variables-and-density-estimation}}

\hypertarget{data}{%
\subsection{Data on perceived risks}\label{data}}

The data used for this paper is from the core module of the Survey of Consumer Expectation(SCE) conducted by the New York Fed, a monthly online survey for a rotating panel of around 1,300 household heads over the period from June 2013 to July 2021, over a total of 97 months.


I primarily rely upon the density forecast of individual earnings by each respondent in the survey to estimate perceived income risks. In particular, the main question used is
framed as the following: ``Suppose that 12 months from now, you are working in the
exact same {[}``main'' if Q11\textgreater1{]} job at the same place you currently work and working the exact same number of hours. In your view, what would you say is the percentage chance that 12 months from now: increased by x\% or more?''.\footnote{As a special feature of the online questionnaire, the survey only moves on to the next question if the probabilities filled in all bins add up to one. This ensures the basic probabilistic consistency of the answers
crucial for any further analysis.} Then, I fit the bin-based density forecast in each survey response with a parametric distribution. \footnote{This follows the approach by \cite{engelberg_comparing_2009} and the researchers in the New York Fed (\cite{armantier_overview_2017}). Appendix \ref{density-estimation-and-variables} documents in details the estimation methodology and its robustness.} The variance of the estimated distribution naturally reveals an individual-specific perceived risk.

Crucially, as the survey question regards the expected earning growth conditional on the same job position, same hours, and the same location, it can be clearly interpreted as the wage. It becomes immediately clear that wage risk only constitutes a part of income risks. This has two important implications. 

First, focusing on the wage risks avoids the problem of confusing earning changes due to voluntary labor supply decisions as risks. Empirical work estimating income risks is often based on data from total earnings or total household income, in which voluntary labor supply decisions inevitably confound the true degree of uninsured idiosyncratic risks. As it is clear regarding the wage, this survey-based measure used here is not subject to this problem. Meanwhile, however, the wage risk also excludes important sources of income risks such as unemployment and job switching. Such major transitions in job career, as some existing research (for instance, \cite{low2010wage}) has shown, are oftentimes the dominant source of income risks facing individual workers. I separately examine unemployment/separation expectations, both of which are surveyed in SCE as well in Section \ref{unemployment-risk-perceptions}. 

%Unemployment and other involuntary job separations are undoubtedly important sources of income risks, but I choose to focus on the same-job/hour earning with the recognition that individuals' income expectations, if any, may be easier to be formed for the current job/hour than when taking into account unemployment risks. Given the focus of this paper being subjective perceptions, this serves as a useful benchmark. What is more assuring is that the bias due to omission of unemployment risk is unambiguous. We could interpret the moments of same-job-hour earning growth as an upper bound for the level of growth rate and a lower bound for the income risk. To put it in another way, the expected earning growth conditional on current employment is higher than the unconditional one, and the conditional earning risk is lower than the unconditional one. At the same time, since SCE separately elicits the perceived probability of losing the current job for each respondent, I could adjust the measured labor income moments taking into account the unemployment risk.


\subsection{Wage data}

I use longitudinal data on individual labor earnings from the 2014-2017 and 2018-2020 panels of the Survey of Income and Program Participation (SIPP)\footnote{Other recent work that estimates income risks using SIPP includes \cite{bayer2019precautionary}. Different from this paper, they use quarterly total household income, instead of the monthly job-specific earning of individuals.}. Each panel of the SIPP is designed to be a nationally representative sample of the U.S. population and surveys thousands of workers. The interviews are conducted once a year to collect the individual's monthly earnings and labor market activity\footnote{This causes the ``seam'' issue well documented in the survey literature\citep{moore2008seam}., which states that cross-wave transitions are systematically larger in magnitudes than within-wave changes. Therefore, I exclude the December-to-January earning growth in estimations to address this issue.}. On average, each individual is surveyed for 33 months over the multiple waves of the survey.

For the purpose of this paper, there are obvious advantages with using SIPP over another commonly used dataset for income risk estimation, the most notable of which is the Panel Study of Income Dynamics (PSID). SIPP surveys monthly labor outcomes of workers such as earnings, hours of work, other detailed records of job transitions and unique employer identifier, while PSID only provides biennial records of labor income for years since 1997. For the  overlapping periods between SIPP and SCE, it is possible to make a direct comparison between realized wage risks at the annual frequency and the ex-ante perceptions of the wage risks. This is particularly crucial if wage risks are time-varying and dependent upon macroeconomic conditions. 

For an apple-to-apple comparison, I obtain the hourly wage of workers with the same employer by dividing the total monthly earnings from the \emph{primary job} by the average hours of work for the same job for those who only stay with the same employer for at least 2 years. I follow the same approach as in \cite{low2010wage} to identify job stayers. In addition, I impose the following criteria. (1) only working-age population between 25-65. (2) only private-sector jobs, excluding workers from government or other public sectors. (3) no days away from work during the reference month without the pay. (4) the same job as the last year. (5) monthly wage rates that are greater than 10 times or smaller than 0.1 times of the average wage are excluded. This leaves me with a monthly panel of 350-1000 individual earners for the sample period 2013m3-2019m12. Appendix \ref{appendix:sipp_data} discusses the data selection and summary statistics in greater details. 